\documentclass[12pt]{article}

\usepackage{bm}
\usepackage{amsfonts}
\usepackage{mathtools}

\begin{document}

\title{Experimental Design}
\author{Shiro Takagi}
\date{2021/6/5 -}
\maketitle

\section{Overview}
My hypothesis is that
\begin{itemize}
    \item I can transform the task-specific short-term memory to general and robust memory if I
    \begin{itemize}
        \item encode temporal information
        \item enforce agents not to use long-term memory directly
        \item protect long-term memory by changing network
    \end{itemize}
\end{itemize}
And, I expect that language manipulation may help to preserve the long-term memory.

I will study if this hypothesis is valid or not. To evaluate the hypothesis above, I have to 
\begin{itemize}
    \item measure the generality of the memory
    \item measure the robustness of the memory
    \item design a way to encode temporal information
    \item design a way to construct abstract semantics and relation by memory
    \item design a way to have agents to use that information
    \item design a way to change network to protect memory
    \item study the influence of the factors above on these measure
    \item design experiments which reflect all the requirements above
\end{itemize}

\section{Generality of Memory}
Our daily experience is just a set of sensory information and the reaction of our internal state to them. 
However, our memory (especially episodic memory) does not seem to be in such a form. Rather, it is like, say 
``I saw a cat yesterday''. This is symbolic and far away from just a collection of sensory inputs. Although 
memory itself is not purely symbolic, it is tightly connected with abstract interpretation of its raw experience.
I call this kind of memory general in that the memory is not unique to a specific memory.

Therefore, it is difficult to directly measure the absolute generality of a memory. Instead, I measure the 
generality indirectly. I hypothesize that a general memory is easier to be ``used'' in more experiences. In other 
words, it has more similarity/overlap with various experiences. We can measure ``how many times a memory is used for other experiences'' 
or ``how many similar memories it has''.

A further indirect measure of the generality of a memory is to measure the generalization of a reinforcement learning agents. 
This is because a general memory can help agents to solve more tasks. This measurement is task-specific and is limited to 
reinforcement learning but common and interpretable.

\section{Robustness of memory}
I say a memory is robust when the memory is hard to be forgotten. In hypothesis.tex, I define memory as 
$\bm{\theta} \in \Theta \subset \mathbb{R}^p$. In that sense, every memory changes even by a small bit of 
perturbation $\bm{\epsilon}$ since $\bm{\theta} \neq \bm{\theta} + \bm{\epsilon}$. This is not the memory 
we are interested in. Rather, we are interested in whether some information $f^I(\bm{\theta})$ extracted from a memory changes or not 
by a perturbation $\bm{\epsilon}$: we say a memory is robust when $f^I(\bm{\theta}) = f^I(\bm{\theta}+ \bm{\epsilon})$.

Hence, we have to define information function $f^I: \Theta \to \mathbb{R}^d$. 

\end{document}